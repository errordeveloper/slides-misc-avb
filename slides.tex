\documentclass{beamer}

\mode<presentation>
{
  \usetheme{Warsaw}
  \setbeamercovered{transparent}
}

\usepackage[english]{babel}
\usepackage[utf8]{inputenc}
\usepackage{times}
\usepackage[T1]{fontenc}
\usepackage{latexsym}

\title[AVB]{Audio/Video Bridging on Local Area Networks}

%\subtitle{Overview of latest technology for digital audio streaming on Ethernet networks.}

\subject{Audio Electronics Design}

\author[Dmitrichenko, Ilya]{Ilya Dmitrichenko <errordeveloper@gmail.com>}

\institute[London Metropolitan University]{}

\date[\today]

%%============================%%

\begin{document}

\begin{frame}
	\titlepage
\end{frame}

\begin{frame}{Outline}
	\tableofcontents[pausesections]
\end{frame}

%%============================%%

\section{Motivation}

\subsection{Overview}

\begin{frame}{ Why AVB is needed? }

\begin{itemize}
	\item Designed for real-time streaming of audio and video data
	\item Common standard approved by joint group \newline of manufacturers as part of \emph{\texttt{IEEE P802}}
	\item Cost-effective cabling and hardware technology
	\item Simple to use and and set-up
	\item Robust operation achieved by a combination \newline of specifically designed protocols
\end{itemize}

\emph{ AVB addresses multiple market segments, \newline including professional A/V and consumer entertainment. }

\end{frame}

%%============================%%

\subsection{Other Technologies}

\begin{frame}[allowframebreaks]{ Background Overview }

\emph{Point-to-Point Links:
\begin{itemize}
	\item MADI and AES/EBU
	\item SDI and HDMI
	\item OptoCore
	\item I2S and USB
	\item IEEE P1394
\end{itemize}}

{Both audio and video manufacturers have already developed their own approaches to achieve
similar capabilities and a number of standards exist using specific digital protocols with
specialized cabling and hardware and cannot be combined with each other!}

\break

\emph{Proprietary Solutions:
\begin{itemize}
\item CobraNet
\item EtherSound
\item Aviom
\item REAC
\item AES50
\end{itemize}}

{These audio networking systems are already using Ethernet. \newline
The approach is not uniformal and cross-compatibility \newline
of these systems is a major issue for audio engineers!}

\break

\begin{itemize}
	\item{ Networking protocols \emph{(such as TCP, UDP and IP)} }
	\begin{itemize}
		\item do not implement time synchronization
		\item were designed for more general data transmission
		\item appied on a different scale of deployment
	\end{itemize}

\item It has been generally proven that \newline
	\emph{Ethernet is capable to handle high definition A/V \newline
	with reasonable quality of service (QoS) }
\item Some technologies listed above achived it\newline 
	with various approaches (mostly proprietary)

\break

There certainly is a way of fine-tuning off-the-shelf networking equipment and software to provide
reliable performance for streaming audio from one computer to another using cross-over \emph{\texttt{CAT5}}
connection or a single switch. Also applying a set of fine-tunned QoS and traffic shaping rules on commodity
equipment may in theory provide reasonable performance in a larger LAN, assuming there is no other traffic.

Another set of approaches exists using RTP and RTSP combined with multicasting, however these
are used for compressed and latency-tollerant media streaming in conferencing and entertainment.

% Control protocols is another important subject, which is not in the scope of this presentation.

\end{itemize}

\end{frame}

%%============================%%

\subsection{Introducing A/V Bridging}

\begin{frame}{ How AVB uses the Ethernet network? }

\begin{itemize}
	\item No IP addresses and ports, use streams instead
	\item Distributed precise clocking hierarchy
	\item Pre-allocated resources for streams
	\item Monitoring of throutput and latency
\end{itemize}

\end{frame}

%%============================%%

\section{Implementation Overview}

\subsection{Layers and Protocols}

\begin{frame}{ Underlining Base Standards }

\emph{IEEE 802.1 - Network Layer 1:}

\begin{itemize}

	\item{ \emph{\texttt{802.1Q}} - \emph{Virtual Bridged Local Area Network (VLAN)} }

	\item{ \emph{\texttt{802.1Qav}} - \emph{Time-Sensitive Forwarding and Queuing } }

	\item{ \emph{\texttt{802.1ak}} - \emph{Multiple Registration Protocol } }

	\item{ \emph{\texttt{802.1Qat}} - \emph{Stream Reservation Protocol (SRP) } }

	\item{ \emph{\texttt{802.1AS}} - \emph{Timing and Synchronization \texttt{(P1588-2008)} } }

	\item{ \emph{\texttt{802.1BA}} - \emph{Audio Video Bridging Systems } }

\end{itemize}

\emph{Network Layers 2 and 3:}

\begin{itemize}
	\item{ \emph{\texttt{P1723}} - \emph{Encapsulation Format for A/V Transport (AVBTP) } }
	\item{ \emph{\texttt{P1733}} - \emph{Correlation of RTP timestamps with PTP } }
\end{itemize}

\emph{Other:}

\begin{itemize}
	\item{ \emph{\texttt{P1588}} - \emph{Precision Time Protocol (PTP) } }
	\item{ \emph{\dots digital media format standards} }
\end{itemize}

\end{frame}

%%============================%%

\begin{frame}[allowframebreaks]{ Clock Synchronization }

\begin{itemize}
	\item \emph{ PTP version 2 is an IEEE standard \texttt{P1588-2008} }
	\item \emph{ For AVB use, a subset has been defined under \texttt{802.1AS} }
Unlike other new protocols, \newline PTP is already used in other application areas:

%\emph{Applications of PTP:}
\begin{itemize}
	\item Industrial Control
	\item Test and Measurement Equipment
	\item A/V Streaming and Control
	\item Telecommunications
\end{itemize}

\end{itemize}

\break

{ PTP defines a number of algorithms to find the delay times on the network
and adjust the clock on all devices to the same time base. This clock is then
used to re-align audio and video frames received in data packets.
This functionality needs to be implemented in physical layer hardware and
sync precision of tens of nano seconds is achievable on a network that fully
implements PTP in switches and end-points. A clock hierarchy of master and
slave devices, including boundary and transparent clocks applies no such network. }

\end{frame}

%%============================%%

\begin{frame}{ Virtual Networks (VLANs)}

\emph{\texttt{802.1Q}} is a base for \emph{Virtual Bridged Local Area Networks}.
Fallowing two amendments were made for \emph{A/V Bridging}:
\begin{itemize}
	\item \emph{\texttt{802.Qat}} - \emph{Stream Reservation Protocol}
	\item \emph{\texttt{802.1Qav}} - \emph{Time-Sensitive Forwarding and Queuing}
\end{itemize}

{ Combination of this two protocols allows allows traffic with AVB tag to take 75\% of total
LAN bandwidth and also make sure the buffered packets are synchronized with PTP clock. }

\end{frame}

%%============================%%

\begin{frame}[allowframebreaks]{ Stream Reservation and Traffic Shaping }

\emph{\texttt{802.1Qat}} and \emph{\texttt{802.1Qav}} introduce the concepts of
\begin{itemize}
	\item \emph{streams} and \emph{slots} or \emph{channels}
	\item \emph{talkers} and \emph{listeners}
\end{itemize}

\begin{itemize}

	\item Provide special \emph{admission control} and \emph{priority tagging}

	\item Talkers advertise their stream

	\begin{itemize}
		\item \emph{(64-bit ID) = (48-bit MAC address) + (16-bit stream ID)}
		\item \emph{QoS requirements}
	\end{itemize}

	\item Listeners request to register for streams

	\item \emph{\texttt{802.1Qat}} is a software protocol

\end{itemize}

\break

\texttt{802.1Qav} is designed to:

\begin{itemize}
	\item prevent packet loss in buffers
	\item synchronize queued packets with PTP clock
\end{itemize}

It defines flow control algorithms to provide:

\begin{itemize}
	\item predictable latency of the path and lower the jitter
	\item synchronized forwarding of buffer queues
	\item prevention of packet "bunching"
\end{itemize}

\break

\begin{itemize}
	\item All \emph{A/V Bridges} are required to

	\begin{itemize}
		\item \emph{verify the bandwidth availability (between the end-points)}
		\item \emph{propagate the advertise message (to and from end-points)}
	\end{itemize}

	\item If the bandwidth is insufficient \newline
		$\rightarrow$ the failure is reported
	\item When a listener registers a stream \newline
		$\rightarrow$ the resources are locked down
	\item The resources are unlocked when it de-registers

\end{itemize}

\end{frame}

%%============================%%

\subsection{Audio in The Network}

\begin{frame}[allowframebreaks]{ Channels and Devices }

On AVB network end-point devices are classified as
\begin{itemize}
	\item \texttt{talkers} - advertise streams
	\item \texttt{listeners} - suscribe to recieve streams
\end{itemize}
Then the set of protocol ensure
\begin{itemize}
	\item best-effort traffic throughoutput
	\item synchronization in transmission and on the end-points
\end{itemize}

\break

\begin{itemize}

\item Audio channels are grouped into \emph{streams}
\item Video or audio channels are refered to as \emph{slots}
\item Stream originates from \emph{one talker} and shares \emph{one clock}
\item Routed on second layer in \emph{unicast} or \emph{multicast} mode
\item The channels are not split out of the stream in the network
\item From the network level only the \emph{unique stream ID} is known
\item Higher channel count in one stream reduces the bandwidth

\end{itemize}

\end{frame}

%%============================%%

\begin{frame}{ Media Formats and Control }

\emph{  AVBTP - the Transport Protocol (P1722) }

\begin{itemize}

	\item Various media formats can be used
	\newline \emph{ (such as MPEG/ISO, PCM or AM824 and others) }

	\item Allows for bridging of \emph{\texttt{P802}} and \emph{\texttt{P1394}} networks

	\item Compatibility with IP streaming is facilitated by \emph{\texttt{P1733}} 

\end{itemize}

\emph{ DECC - the higher layer protocol (P1722.1)  }
\begin{itemize}
	\item Discovery
	\item Enumeration
	\item Connection
	\item Control 
\end{itemize}

\end{frame}

%%============================%%

\section{Outlook on Design Perspectives}

\subsection{New Hardware!}

\begin{frame}{ Design Solutions }

\begin{itemize}
	\item National Semiconductors DP83640
	\newline Ethernet Physical Layer Chip with P1588 
	\item XMOS AVB Reference Design Software
	\newline for XS1 Processor Architecture
	\item Marvell Kirwood ARM SoC
	\newline and 88E0000 Yukon \& LinkStreet Ethernet ICs
	\item Freescale MPC831X PPC SoC
	\item Lab-X and Xilinx FPGA cores
\end{itemize}

\end{frame}

%%============================%%

\subsection{What AVB can do?}

\begin{frame}{ Peformance Metrics }

\emph{Source: XMOS reference design documentation}
\begin{itemize}
	\item Network: 100MB Ethernet
	\begin{itemize}
		\item Channels: 32x32 and 16x16
		\item Audio Quality: 48kHz/24-bit and 96kHz/24-bit
	\end{itemize}
	\item Network: 1GB Ethernet
	\begin{itemize}
		\item Channels: 72x72 and 36x36
		\item Audio Quality: 48kHz/24-bit and 96kHz/24-bit
	\end{itemize}
\end{itemize}

\emph{Source: Harman product presentation}
\begin{itemize}
	\item Network: 1GB Ethernet
	\item Latency: <2ms
	\item Channels: 300 x 300
	\item Audio Quality: 48kHz/24-bit
\end{itemize}


\end{frame}

%%============================%%

\begin{frame}{ More Data : Bandwidth of channels per stream }
\begin{table}
\caption{\emph{Source: XMOS reference design documentation}}

\begin{tabular} {| r || c | c | c | } \hline
$\frac{kHz}{Mbps}$ & \emph{48} & \emph{96} & \emph{192} \\ \hline \hline
\emph{7.81} & 2 or 1 &- &- \\ \hline
     \emph{10.88} & 4 & 2 or 1  &- \\ \hline
     \emph{17.02} & 8 & 4 & 2 or 1 \\ \hline
     \emph{29.31} & 16 & 8 & 4 \\ \hline
     \emph{53.89} & 32 & 16 & 8 \\ \hline
\end{tabular}

\end{table}

\end{frame}

%%============================%%

\subsection{\dots More Information}

\begin{frame}{ On-line Reference }

\begin{thebibliography}{10}
%
\beamertemplatearticlebibitems

	\bibitem{AVnu}
	AVnu Alliance
	\newblock{ \em Homepage: }
	\newblock{ \em http://avnu.org/ }

\beamertemplatearticlebibitems

	\bibitem{avbridges}
	IEEE Task Group
	\newblock{ \em Homepage: }
	\newblock{ \em http://ieee802.org/1/pages/avbridges.html }

\beamertemplatearticlebibitems

	\bibitem{xillixipcore}
	Xilinx Ethernet AVB Endpoint IP LogiCORE
	\newblock{ \em Product Page: }
	\newblock{ \em http://xilinx.com/products/ipcenter/DO-DI-EAVB-EPT.htm }

\beamertemplatearticlebibitems

	\bibitem{xmosrefdesign}
	XMOS AVB Refferece Desgin
	\newblock{ \em Application Design for XS1 Processor: }
	\newblock{ \em http://xmos.com/applications/avb }

\end{thebibliography}

\end{frame}

%%============================%%

\end{document}
