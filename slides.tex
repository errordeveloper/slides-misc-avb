{ Why AVB is needed? }

\begin{itemize}
	\item Designed for audio and video real-time straming and control over Ethernet
	\item Common standard approved by joint group of manufacturers as part of IEEE 802
	\item Cost-effective cabling and hardware technology
	\item Simple to use and and set-up
	\item Robust operation achived by a combination of specific prtocols
\end{itemize}

{ Backgorund Overview }

Networking protocols (such as TCP, IP and UDP) do not implement time synctronization and
were designed for more general data transmission on a different scale of deployment.

Both audio and video manufacturers have already developed their own approaches to achive
similar capabilities and a numeber of standards exist using specific digital protocols with
specialized cabling and hardware, but these are point-to-point and cannot be combined with each other
\emph{(Examples: MADI, SDI, HDMI, AES/EBU, OptoCore, I2S, IEEE 1394 and USB)}.
Proprietary networking systems are already using Ethernet, but the approach is not uniformal
and cross-compatibility of these systems is a major issue for audio engineers
\emph{(Examples: CobraNet, EtherSound, Aviom, REAC and AES50)}.
Control protocols is another important subject, which is not in the scope of this presentation.

In general, it has been proven that Ethernet is capable to handle high definition audio and video
with reasonably low latency. Various technolodies listed above have achived this in different ways.

There certainly is a way of fine-tuning off-the-shelf networking equipment and software to provide
realiable preformance for streaming audio from one computer to another using cross-over CAT5 connection
or a single switch. Also applying a set of fine-tunned QoS and trafic shaping rules on comodity
equipment may in theory provide reasonable performance in larger LAN, assuming there is no other traffic.

Another set of approaches exists using RTP and RTSP, as well as multicasting, hower these
are used for compressed and latency-tollerant audio or video streaming in conferencing and media.



{ AVB addresses multiple market segments, including professional A/V and consumer entertainment. }

How AVB uses the ethernet networking?

\begin{itemize}
	\item No IP addresses and ports, use streams instead
	\item Distributed clocking hierarchy
	\item Pre-allocated resources for streams
	\item Monitoring of throutput and latency
\end{itemize}

{ Underlining Base Standards }
IEEE 802.1 - Network Layer-1:

	{ 802.1Q - Virtual Bridged Local Area Network }

	{ 802.1Qav - Forwarding and Queuing Enhancements for Time-Sensitive Streams }

	{ 802.1ak - Multiple Registration Protocol }
	{ 802.1Qat - Stream Reservation Protocol (SRP)}

	{ 802.1AS - Timing and Synchronization (based on P1588)}

	{ 802.1BA - Audio Video Bridging Systems }

Network Layers 2 and 3:
	\item P1722 Encapsulation Format for A/V Transport in AVB (AVBTP)
	\item P1733 Correlation of RTP timestamps with PTP
Other:
	\item P1588 Prescision Time Protocol (PTP)



{ Clock Syncronization }

%{ 802.1AS - Timing and Synchronization for Time-Sensitive Applications in Bridged Local Area Networks }

{ Precision Time Protocol (PTP) version 2 is an IEEE standard P1588-2008.
For AVB use, a subset has been defined under standard 802.1AS.
Unlike other protocols in AVB, PTP is already used in other application areas.}
\begin{itemize}
	\item Applications of PTP:
	\begin{itemize}
		\item Industrial Control
		\item Test and Measurment Equipment
		\item A/V Streaming and Control
		\item Telecommunications
	\end{itemize}
\end{itemize}

{ PTP defines a number of algorithms to find the delay times on the network
and adjust the clock on all devices to the same time base. This clock is then
used to re-alling audio and video frames recieved in data packets.
This functionality needs to be implemented in physical layer hardware and
sync precision of tens of nano seconds is achivable on a network that fully
implements PTP in switches and end-points. A clock hierarchy of master and
slave devices, including boundry and transparent clocks applies no such network. }

{ Virtual Networks }

IEEE defined 802.1Q as a base standard for Virtual Bridged Local Area Networks,
in order to implement A/V Bridgin two ammendments were made to 802.1Q:
\begin{itemize}
	\item 802.Qat - Stream Reservation Protocol
	\item 802.1Qav - Forwarding and Queuing Enhancements for Time-Sensitive Streams
\end{itemize}

Combination of this two protocols allows allows traffic with AVB tag to take 75\% of total
LAN bandwidth and also make sure the buffered packets are syncronized with PTP clock.

{ Stream Reservation and Traffic Shaping }

802.1Qat and 802.1Qav together form the base for concepts of
\item \texttt{streams} and \texttt{channels}
\item \texttt{talkers} and \texttt{listeners}

It also provides admisson control and priority tagging.

In terms of 802.1Qat, talkers advertise their stream ID
\item 48-bit MAC address + 16-bit ID the stream.
Additionally QoS requirenments are advertised.
All bridges are required to
\item verify the bandwidth avaliablity, and
\item propagate the advertise message.
If the bandwidth is insuficient, the failure is reported.
When a listener registers a stream, the resources are locked down.
The resources are unlocked when it de-registers.

802.1Qav is designed to preven packet loss in buffers and syncronize with PTP.
It defines flow control algorithms to provide:
\item predictable latency of the path and lower the jitter
\item syncronized forwarding of buffer queues
\item prevention of "bunching"

%%============================%%

{ Channels and Devices }

On AVB network end-point devices are classified as
\begin{itemize}
	\item \texttt{talkers} - advertise stream ID
	\item \texttt{listeners} - register to recieve streams
\begin{itemize}

Audio channels are grouped into streams
Combination of video and audio frames in the streams are groupped as slots
Stream originates from one \texttt{talker} and has the same clock
Streams are routed on Ethernet L2 in unicast or multicast mode
The channels are not split out of the stream on the network level
From the network level only a stream ID (64-bit) is known
Higher channel count in one stream reduces the bandwidth usage

{  AVBTP - the Transport Protocol (IEEE P1722) }


Various media formats can be used
(such as MPEG (ISO), PCM or AM824 and others)

{ Control and Mangement }

DECC (IEEE P1722.1) is the higher layer protocol covering
\begin{itemize}
	\item device discovery
	\item device enumeration
	\item connection management
	\item device control 
\end{itemize}

%%============================%%

\begin{itemize}
	\item Design Solutions
	\begin{itemize}
		\item National Semiconductors DP83640 Ethernet Physical Layer Chip with P1588 
		\item XMOS AVB Reference Design Software
		\item Marvell Kirwood ARM SoC and 88E0000 Yukon \& LinkStreet Ethernet ICs
		\item Freescale MPC831X PPC SoC
		\item Lab-X and Xilinx FPGA cores
	\end{itemize}
\end{itemize}

%%============================%%

{ Peformance Metrics }
{Source: XMOS reference design documentation}
\begin{itemize}
	\item Network: 100MB Ethernet
	\begin{itemize}
		\item Channels: 32x32 and 16x16
		\item Audio Quality: 48kHz/24-bit and 96kHz/24-bit
	\end{itemize}
	\item Network: 1GB Ethernet
	\begin{itemize}
		\item Channels: 72x72 and 36x36
		\item Audio Quality: 48kHz/24-bit and 96kHz/24-bit
	\end{itemize}
\end{itemize}


{Source: Harman product presentation}
\begin{itemize}
	\item Network: 1GB Ethernet
	\item Latency: <2ms
	\item Channels: 300 x 300
	\item Audio Quality: 48kHz/24-bit
\end{itemize}

{ Bandwidth of channels per stream }
\begin{table}
\caption{Source: XMOS reference design documentation}
\begin{tabular} { r | c | c | c | }
$\fract{\tt{SR} kHz}{\tt{BW} Mbps}$ & 48 & 96 & 192 \\ \hline
			7.81        & 2 or 1  &-   &- \\ \hline
			10.88       & 4  & 2 or 1  &- \\ \hline
			17.02       & 8  & 4  & 2 or 1 \\ \hline
			29.31       & 16 & 8  & 4 \\ \hline
			53.89       & 32 & 16 & 8 \\ \hline
\end{tabular}
