{ Why AVB is needed? }

\begin{itemize}
	\item Designed for audio and video real-time straming and control over Ethernet
	\item Common standard approved by joint group manufacturers and IETF
	\item Cost-effective cabling and hardware technology
	\item Simple to use and and set-up
	\item Robust operation achived by a combination of specific prtocols
\end{itemize}

{ Backgorund Overview }

Networking protocols (such as TCP, IP and UDP) do not implement time synctronization and
were designed for more general data transmission on a different scale of deployment.

Both audio and video manufacturers have already developed their own approaches to achive
similar capabilities and a numeber of standards exist using specific digital protocols with
specialized cabling and hardware, but these are point-to-point and cannot be combined with each other
\emph{(Examples: MADI, SDI, HDMI, AES/EBU, OptoCore, I2S, IEEE 1394 and USB)}.
Proprietary networking systems are already using Ethernet, but the approach is not uniformal
and cross-compatibility of these systems is a major issue for audio engineers
\emph{(Examples: CobraNet, EtherSound, Aviom, REAC and AES50)}.
Control protocols is another important subject, which is not in the scope of this presentation.

In general, it has been proven that Ethernet is capable to handle high definition audio and video
with reasonably low latency. Various technolodies listed above have achived this in different ways.

There certainly is a way of fine-tuning off-the-shelf networking equipment and software to provide
realiable preformance for streaming audio from one computer to another using cross-over CAT5 connection
or a single switch. Also applying a set of fine-tunned QoS and trafic shaping rules on comodity
equipment may in theory provide reasonable performance in larger LAN, assuming there is no other traffic.

Another set of approaches exists using RTP and RTSP, as well as multicasting, hower these
are used for compressed and latency-tollerant audio or video streaming in conferencing and media.



{ AVB addresses multiple market segments, including professional A/V and consumer entertainment. }

How AVB uses the ethernet networking?

\begin{itemize}
	\item No IP addresses and ports, use streams instead
	\item Distributed clocking hierarchy
	\item Pre-allocated resources for streams
	\item Monitoring of throutput and latency
\end{itemize}

{ Underlining Base Standards }

	\item P1722 The AVB Transport Protocol
	\time P1588 (802.1AS)

\item \dots

802.1Q Virtual Local Area Networks

% Precision Time Protocol (version 2)

{ Precision Time Protocol (PTP) version 2 is an IEEE standard 1588-2008.
For AVB use, a subset has been defined under IETF standard 802.1AS.
Unlike other protocols in AVB, PTP is already used in other application areas.}
\begin{itemize}
	\item Applications of PTP:
	\begin{itemize}
		\item Industrial Control
		\item Test and Measurment Equipment
		\item A/V Streaming and Control
		\item Telecommunications
	\end{itemize}
\end{itemize}

{ PTP defines a number of algorithms to find the delay times on the network
and adjust the clock on all devices to the same time base. This clock is then
used to re-alling audio and video frames recieved in data packets.
This functionality needs to be implemented in physical layer hardware and
sync precision of tens of nano seconds is achivable on a network that fully
implements PTP in switches and end-points. A clock hierarchy of master and
slave devices, including boundry and transparent clocks applies no such network. }

{ IEEE 1588 Standard for a Precision Clock Synchronization Protocol for Networked Measurement and Control Systems (also known at PTP). }
{ IEEE Layer 2 Transport Protocol Working Group for Time-Sensitve Streams. }
{ IEEE P1722.1 defines the higher layer protocol for IEEE P1722 based devices.
This standard covers device discovery, device enumeration, connection management
and device control protocols for finding and connecting IEEE 1722 based devices. }


%%============================%%

\begin{itemize}
	\item Design Solutions
	\begin{itemize}
		\item National Semiconductors DP83640 Ethernet Physical Layer Chip with P1588 
		\item XMOS AVB Reference Design Software
		\item Marvell Kirwood ARM SoC and 88E0000 Yukon \& LinkStreet Ethernet ICs
		\item Freescale MPC831X PPC SoC
		\item Lab-X and Xilinx FPGA cores
	\end{itemize}
\end{itemize}

%%============================%%

{ Peformance Metrics }
\item Network: 100MB Ethernet
\item Latency: >2ms
\item Channels: ?
\item Audio Quality: 96k/24-bit ?

\item Network: 1GB Ethernet
\item Latency: >2ms
\item Channels: 300 x 300
\item Audio Quality: 48kHz/24-bit
