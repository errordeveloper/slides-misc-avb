% $Header: /cvsroot/latex-beamer/latex-beamer/solutions/conference-talks/conference-ornate-20min.en.tex,v 1.7 2007/01/28 20:48:23 tantau Exp $

\documentclass{beamer}

% This file is a solution template for:

% - Talk at a conference/colloquium.
% - Talk length is about 20min.
% - Style is ornate.



% Copyright 2004 by Till Tantau <tantau@users.sourceforge.net>.
%
% In principle, this file can be redistributed and/or modified under
% the terms of the GNU Public License, version 2.
%
% However, this file is supposed to be a template to be modified
% for your own needs. For this reason, if you use this file as a
% template and not specifically distribute it as part of a another
% package/program, I grant the extra permission to freely copy and
% modify this file as you see fit and even to delete this copyright
% notice. 


\mode<presentation>
{
  \usetheme{Warsaw}
  % or ...

  \setbeamercovered{transparent}
  % or whatever (possibly just delete it)
}


\usepackage[english]{babel}
% or whatever

\usepackage[latin1]{inputenc}
% or whatever

\usepackage{times}
\usepackage[T1]{fontenc}
\usepackage{latexsym}
% Or whatever. Note that the encoding and the font should match. If T1
% does not look nice, try deleting the line with the fontenc.


\title[WMI] % (optional, use only with long paper titles)
{WMI - Wireless Music Instruments}

\subtitle
{BSc Final Year Project}

\author[Dmitrichenko, Ilya] % (optional, use only with lots of authors)
{Ilya Dmitrichenko <errordeveloper@gmail.com>}
%{F.~Author\inst{1} \and S.~Another\inst{2}}
% - Give the names in the same order as the appear in the paper.
% - Use the \inst{?} command only if the authors have different
%   affiliation.

%\institute[Universities of Somewhere and Elsewhere] % (optional, but mostly needed)
%{
%  \inst{1}%
%  Department of Computer Science\\
%  University of Somewhere
%  \and
%  \inst{2}%
%  Department of Theoretical Philosophy\\
%  University of Elsewhere}
%% - Use the \inst command only if there are several affiliations.
%% - Keep it simple, no one is interested in your street address.

\date[November 2010] % (optional, should be abbreviation of conference name)
%{Conference on Fabulous Presentations, 2003}
% - Either use conference name or its abbreviation.
% - Not really informative to the audience, more for people (including
%   yourself) who are reading the slides online

\subject{Embedded Electronics Engineering Project Presentation}
% This is only inserted into the PDF information catalog. Can be left out. 



% If you have a file called "university-logo-filename.xxx", where xxx
% is a graphic format that can be processed by latex or pdflatex,
% resp., then you can add a logo as follows:

% \pgfdeclareimage[height=0.5cm]{university-logo}{university-logo-filename}
% \logo{\pgfuseimage{university-logo}}



% Delete this, if you do not want the table of contents to pop up at
% the beginning of each subsection:
%%\AtBeginSubsection[]
%%{
%%  \begin{frame}<beamer>{Outline}
%%    \tableofcontents[currentsection,currentsubsection]
%%  \end{frame}
%%}


% If you wish to uncover everything in a step-wise fashion, uncomment
% the following command: 

%\beamerdefaultoverlayspecification{<+->}


\begin{document}

\begin{frame}
  \titlepage
\end{frame}

\begin{frame}{Outline}
  \tableofcontents[pausesections]
  % You might wish to add the option [pausesections]
\end{frame}


% Structuring a talk is a difficult task and the following structure
% may not be suitable. Here are some rules that apply for this
% solution: 

% - Exactly two or three sections (other than the summary).
% - At *most* three subsections per section.
% - Talk about 30s to 2min per frame. So there should be between about
%   15 and 30 frames, all told.

% - A conference audience is likely to know very little of what you
%   are going to talk about. So *simplify*!
% - In a 20min talk, getting the main ideas across is hard
%   enough. Leave out details, even if it means being less precise than
%   you think necessary.
% - If you omit details that are vital to the proof/implementation,
%   just say so once. Everybody will be happy with that.

\section{Motivation}

\subsection{Main Application Area}

\begin{frame}{Main Application Area}
\begin{itemize}

	\item{IP WSN for Music Performance \& Recording}

	\begin{itemize}

	\item Why WSN?
	\item Why IP?

	\end{itemize}

	\item{Constrains:}

	\begin{itemize}
		
		\item Deployability and Support
		\item Performance and Robustness

	\end{itemize}

	\item{Commercial and Practical Interests}

\end{itemize}
\end{frame}

\subsection{Additional Aspects \& Tasks}
\begin{frame}{Additional Aspects \& Tasks}

\begin{itemize}
	
	\item Customised Linux OS

	\item Facility scripts and libraries

	\item DSP algorithms

	\item MIDI over WPAN

	\item WSN for audio measurements

\end{itemize}
\end{frame}

\section{Design and Implementation}

\subsection{Hardware Element} 
\begin{frame}{Hardware Element}

\begin{itemize}

	\item Latest low-power microcontroller with integrated radio

	\item Low-coast development system

	\item Robust interconnection design

	\item Various sensors to be tested for suitability

\end{itemize}
\end{frame}

\subsection{Software Element}
\begin{frame}{Software Element}

\begin{itemize}

	\item Utilise open source free RTOS

	\item Test all and use the best

	\item Port the driver code for the transceiver
	
	\item Board-specific abstraction and sensor drivers

\end{itemize}
\end{frame}

\subsection{Project Work Plan}

\begin{frame}{Roadmap}

\begin{itemize}

	\item Development boards and software testing \pause
	\item Publish the code on the website
	\newline after the first tests are successful \pause

	\item Driver code is due by the interim report \pause

	\item Application code and DSP system is due in the final stage

\end{itemize}
\end{frame}


\section{Summary \& References}

\subsection{Scope Summary}
\begin{frame}{Scope Summary}

  % Keep the summary *very short*.
  \begin{itemize}
   \pause
  \item
    This is quite a \alert{challenging project}
   \pause
  \item
    It involves \alert{various aspects}
   \newline spanning multiple areas in embedded system design
   \pause
  \item
    There is a number of concerns and conclusions
    \newline to be made upon competition of \alert{the experiment}
  \end{itemize}

  \pause
  
  % The following outlook is optional.
  \vskip0pt plus.5fill
  \begin{itemize}
  \item
    \emph{Commitment}
    \begin{itemize}
    \item
      \emph{Experiments and Tests}
    \item
      \emph{Contribute to the "Open Source"}
    \end{itemize}

\end{itemize}
\end{frame}



\subsection*{For Further Information}

\begin{frame}{For Further Information}
    
\begin{thebibliography}{10}
    
  \beamertemplatearticlebibitems

  \bibitem{ProjectWebsite}
    I.~Dmitrichenko
    \newblock {\em Wireless Music Instruments}
    \newblock {\em Final Year Project Website}
    \newblock {\em http://wmi.new-synth.info/}
 
    
\end{thebibliography}
\end{frame}

\end{document}
